\documentclass[12pt]{article}
\usepackage{amsmath}
\usepackage{amsfonts}
\usepackage{float}
\usepackage{tikz}
\usepackage{graphicx}
\usepackage{enumerate}
\usepackage{mathrsfs}
\usepackage{adjustbox}
\usetikzlibrary{arrows.meta, automata, positioning,decorations.pathreplacing, decorations.markings}

\title{MATH381 Assignment 2}
\author{Ben Vickers}
\date{Due: 11:55 PM, Thursday 5 September 2024}

\begin{document}

\maketitle

\noindent \textbf{Question 1 (4 marks)} Prove using the sequential characterisation of closed sets that the intersection of an \textit{arbitrary} collection of closed subsets of \( \mathbb{C}\) is again closed. That is, prove that if \(F_j \left(j \in J\right)\) are closed subsets of \(\mathbb{C}\) where \(J\) is an arbitrary set of indices, then \(\bigcap_{j \in J} F_j\) is closed.\newline

\noindent \textbf{Solution:}
\[
\text{Set }F = \bigcap_{j \in J} F_j = F_1 \cap F_2 \cap \cdots \cap F_m
\]
\noindent The specifity of the word arbitrary in relation to the collection of subsets implies $m$ could be infinite. I.e. $J$ is any arbitrary set of indices that is not necessarily finite. \newline
\linebreak
To prove that $F$ is closed using the sequential characterisation of closed sets, we can take a sequence, \( \left(z_n\right)_{n=1}^{\infty} \in F\) such that \(z_n \to z \in \mathbb{C}\). We aim to prove that \(z \in F\). \newline
\linebreak
We know \(\forall \, j \in J\)  :
\begin{itemize}
    \item \(z_n \in F_j \) as \(z_n \in F = \bigcap_{j \in J} F_j \)
    \item \(F_j \) is closed (as per the question)
\end{itemize}
It then follows from the sequential characterisation of closed sets that as \(z_n \) is convergent to some \(z \in \mathbb{C}\), \(z_n \in F_j \,\left(\forall j \in J\right)\) and \(F_j\) is closed that \(z \in F_j\). \newline
\linebreak
Then by the properties of the intersection of an arbitrary collection of subsets of \(\mathbb{C}\), as \(z \in F_j \,\, \left(\forall j \in J\right)\), we have \(z \in F =  \bigcap_{j \in J} F_j\)\newline
\linebreak
As the following conditions have now been shown to hold: \( \textbf{1)} \, z_n \to z \in \mathbb{C}\quad \textbf{2)}\, z_n \in F \quad \textbf{3)}\, z \in F  \), it follows from the sequential characterisation of closed sets that \(\bigcap_{j \in J} F_j \, \left(= F\right)\) is closed. \(\quad \Box\)\newline
\linebreak
\noindent \textbf{Question 2 (5 marks)} Let \( f(z) := ln|z| + iArg(z)\), where ln \( : (0, \infty) \to \mathbb{R}\) is the natural logarithm and \(Arg(z)\) is the principal argument of \(z\). Prove that \(f\) satisfies the Cauchy-Riemann equations on the open upper half-plane.
\[
U := \{z: \text{Im}\,z > 0\}\quad \text{which is identified with} \quad \{\left(x,y\right) \in \mathbb{R}^2:y > 0\}
\]
\noindent by writing Arg\(\left(x + iy\right)\) as arccot\(\left(\frac{x}{y}\right)\) when \(y > 0\). Recall that the inverse cotangent, arcot is a differentiable function from \(\mathbb{R}\) into \(\left(0,\pi\right)\) with arccot\(^{\prime}\)(t) = \(-\frac{1}{1+t^2}\) for \(t \in \mathbb{R}\).\newline
\linebreak
\textbf{Solution:} \newline 
\linebreak
We consider the open upper half-plane, \(U := \{z: \text{Im}\,z > 0\}\) as three regions. One region either side of the imaginary axis and another on the imaginary axis. All with imaginary parts greater than 0.\newline
\[
\text{I.e. }U = L \cup I \cup R \text{ where:}
\]
\begin{itemize}
    \item  \(L := \{\left(x,y\right): y > 0, x < 0\}\)
    \item \(R  := \{\left(x,y\right): y > 0, x > 0\}\)
    \item \(I := \{\left(x,y\right): y > 0, x =0\}\)
\end{itemize}
\[\]
\(\textit{Case 1: } \text{on the open upper left half-plane, } L := \{\left(x,y\right): y > 0, x < 0\}\)
\begin{center}
\resizebox{0.8\textwidth}{!}{%
    \begin{tikzpicture}
        % Draw the axes
        \draw[->] (-4,0) -- (2,0) node[below right] {$\text{Re}(z)$};
        \draw[->] (0,-1) -- (0,3) node[above left] {$\text{Im}(z)$};
        
        % Draw the grid lines
        \draw[dotted] (-4,2) -- (2,2);
        \draw[dotted] (-2,-1) -- (-2,3);
        
        % Draw the point z
        \coordinate (z) at (-2,2);
        
        % Draw the blue line from (0,0) to z
        \draw[blue, thick] (0,0) -- (z);
    
        % Draw the point z and labels
        \fill (z) circle (3pt);
        \node at (-2,2) [above left] {$z$};
        
        % Draw the dotted lines from the point z
        %\draw[dotted] (-2,0) -- (-2,2);
       % \draw[dotted] (0,2) -- (-2,2);
        
        % Add labels for x and y coordinates
        \node at (-2,-0.3) {$x$};
        \node at (0.3,2) {$y$};
    
        % Draw the angle theta
        \draw[blue, thick] (0.5,0) arc[start angle=0,end angle=135,radius=0.5cm];
        \node[blue] at (0.5,0.5) {$\theta$};
        \draw[green, thick] (-0.5,0) arc[start angle=180,end angle=135,radius=0.5cm];
        \node[green] at (-0.7,0.2) {$\alpha$};

    \end{tikzpicture}
    }
\end{center}

\[
\cot\left(\alpha\right) = \frac{-x}{y}
\]
\[
\implies \alpha = cot^{-1}\left(\frac{-x}{y}\right)
\]
\[
\theta = \text{Arg} \, z = \pi - \alpha = \pi - cot^{-1}\left(\frac{-x}{y}\right) = cot^{-1}\left(\frac{x}{y}\right)
\]
\[
\text{So we now have: } f(z) = ln|z| + i\,Arg\left(z\right) = ln\left(\sqrt{x^2 + y^2}\right) + i\,arccot\left(\frac{x}{y}\right)
\]
\[
\text{[Notice } |z| \subset \left(0,\infty\right) \text{ as } y > 0 \text{ so the domain of } ln \text{ is satisfied.]}
\]
\[
f(z) = Re\left(f(z)\right) + i \cdot Im\left(f(z)\right) = u(x,y) + i \cdot v(x,y)
\]
\[
\text{So we have }u(x,y) = ln\left(\sqrt{x^2 + y^2}\right) \text{ and } v(x,y) = arccot\left(\frac{x}{y}\right)
\]
Now differentiating the real and imaginary parts partially with respect to both \(x\) and \(y\) using the chain rule and given derivative of \(arccot\), we obtain:
\[
\frac{\partial u(x,y)}{\partial x} = \frac{\frac{1}{2} \cdot 2x\cdot \left(x^2+y^2\right)^{\frac{-1}{2}}}{\sqrt{x^2+y^2}} = \frac{x}{x^2+y^2}
\]
\[
\frac{\partial u(x,y)}{\partial y} = \frac{\frac{1}{2} \cdot 2y\cdot \left(x^2+y^2\right)^{\frac{-1}{2}}}{\sqrt{x^2+y^2}} = \frac{y}{x^2+y^2}
\]
\[
\frac{\partial v(x,y)}{\partial x} = \frac{-1}{1 + \left(\frac{x}{y}\right)^2}\cdot \left(\frac{1}{y}\right) = \frac{-y}{x^2+y^2}
\]
\[
\frac{\partial v(x,y)}{\partial y} = \frac{-1}{1 + \left(\frac{x}{y}\right)^2}\cdot \left(\frac{-x}{y^2}\right) = \frac{x}{x^2+y^2}
\]
So the Cauchy-Riemann equations:
\[
\frac{\partial u}{\partial x} = \frac{\partial v}{\partial y} \quad \text{and} \quad \frac{\partial u}{\partial y} = -\frac{\partial v}{\partial x}
\]
have been satisfied by \(f\) on the open upper left half-plane.
\[\]
\(\textit{Case 2: } \text{on the open upper right half-plane, } R := \{\left(x,y\right): y > 0, x > 0\}\)
\begin{center}
\resizebox{0.8\textwidth}{!}{%
    \begin{tikzpicture}
        % Draw the axes
        \draw[->] (-2,0) -- (4,0) node[below right] {$\text{Re}(z)$};
        \draw[->] (0,-1) -- (0,3) node[above left] {$\text{Im}(z)$};
        
        % Draw the grid lines
        \draw[dotted] (-2,2) -- (4,2);
        \draw[dotted] (2,-1) -- (2,3);
        
        % Draw the point z
        \coordinate (z) at (2,2);
        
        % Draw the blue line from (0,0) to z
        \draw[blue, thick] (0,0) -- (z);
    
        % Draw the point z and labels
        \fill (z) circle (3pt);
        \node at (2,2) [above right] {$z$};
        
        % Draw the dotted lines from the point z
        %\draw[dotted] (2,0) -- (2,2);
        %\draw[dotted] (0,2) -- (-2,2);
        
        % Add labels for x and y coordinates
        \node at (2,-0.3) {$x$};
        \node at (-0.3,2) {$y$};
    
        % Draw the angle theta
        \draw[blue, thick] (0.5,0) arc[start angle=0,end angle=45,radius=0.5cm];
        \node[blue] at (0.7,0.2) {$\theta$};

    \end{tikzpicture}
    }
\end{center}

\[
\cot\left(\theta\right) = \frac{x}{y}
\]
\[
\implies \theta = \text{Arg} \, z = cot^{-1}\left(\frac{x}{y}\right)
\]
\[
\text{So we now have: } f(z) = ln|z| + i\,Arg\left(z\right) = ln\left(\sqrt{x^2 + y^2}\right) + i\,arccot\left(\frac{x}{y}\right)
\]
\[
\text{[Notice } |z| \subset \left(0,\infty\right) \text{ as } y > 0 \text{ so the domain of } ln \text{ is satisfied.]}
\]
\[
f(z) = Re\left(f(z)\right) + i \cdot Im\left(f(z)\right) = u(x,y) + i \cdot v(x,y)
\]
\[
\text{So we have }u(x,y) = ln\left(\sqrt{x^2 + y^2}\right) \text{ and } v(x,y) = arccot\left(\frac{x}{y}\right)
\]

\noindent Now differentiating the real and imaginary parts partially with respect to both \(x\) and \(y\) using the chain rule and given derivative of \(arccot\), we obtain:
\[
\frac{\partial u(x,y)}{\partial x} = \frac{\frac{1}{2} \cdot 2x\cdot \left(x^2+y^2\right)^{\frac{-1}{2}}}{\sqrt{x^2+y^2}} = \frac{x}{x^2+y^2}
\]

\[
\frac{\partial u(x,y)}{\partial y} = \frac{\frac{1}{2} \cdot 2y\cdot \left(x^2+y^2\right)^{\frac{-1}{2}}}{\sqrt{x^2+y^2}} = \frac{y}{x^2+y^2}
\]

\[
\frac{\partial v(x,y)}{\partial x} = \frac{-1}{1 + \left(\frac{x}{y}\right)^2}\cdot \left(\frac{1}{y}\right) = \frac{-y}{x^2+y^2}
\]

\[
\frac{\partial v(x,y)}{\partial y} = \frac{-1}{1 + \left(\frac{x}{y}\right)^2}\cdot \left(\frac{-x}{y^2}\right) = \frac{x}{x^2+y^2}
\]

So the Cauchy-Riemann equations:

\[
\frac{\partial u}{\partial x} = \frac{\partial v}{\partial y} \quad \text{and} \quad \frac{\partial u}{\partial y} = -\frac{\partial v}{\partial x}
\]

have been satisfied by \(f\) on the open upper right half-plane.\newline
\linebreak
\(\textit{Case 3: } \text{The imaginary axis for \(y>0\), } I := \{\left(x,y\right): y > 0, x =0\}\)
\begin{center}
\resizebox{0.6\textwidth}{!}{%
    \begin{tikzpicture}
        % Draw the axes
        \draw[->] (-2,0) -- (2,0) node[below right] {$\text{Re}(z)$};
        \draw[->] (0,-2) -- (0,5) node[above left] {$\text{Im}(z)$};
        
        % Draw the grid lines
        \draw[dotted] (-2,3) -- (2,3);
        \draw[dotted] (0,-2) -- (0,3);
        
        % Draw the point z
        \coordinate (z) at (0,3);
        
        % Draw the blue line from (0,0) to z
        \draw[blue, thick] (0,0) -- (z);
    
        % Draw the point z and labels
        \fill (z) circle (3pt);
        \node at (0,3) [above right] {$z$};
        
        % Draw the dotted lines from the point z
        %\draw[dotted] (2,0) -- (2,2);
        %\draw[dotted] (0,2) -- (-2,2);
        
        % Add labels for x and y coordinates
        \node at (0,-0.3) {$x$};
        \node at (-0.3,3) {$y$};
    
        % Draw the angle theta
        \draw[blue, thick] (0.5,0) arc[start angle=0,end angle=90,radius=0.5cm];
        \node[blue] at (0.5,0.5) {$\theta$};
    \end{tikzpicture}
    }
\end{center}

\noindent We have already shown that for all \(y > 0\), if \(x<0\) or \(x>0\) we have:
\[
    f(z) = ln|z| + i\,Arg\left(z\right) = ln\left(\sqrt{x^2 + y^2}\right) + i\,arccot\left(\frac{x}{y}\right)
\]
\noindent So although at \(x=0\) we have:
\[
\text{Arg}z = \theta = \cot^{-1} \left(\frac{x}{y}\right) = \cot^{-1}\left(0\right) = \frac{\pi}{2}
\]
\noindent We need still need to consider Arg\(z = arccot\left(\frac{x}{y}\right)\) to allow for differentiating with respect to \(x\). As when differentiating with respect to \(x\) we need to allow for changes in \(x\) around the neighbourhood of \(x=0\).\newline
\linebreak

\noindent Note this holds as we have now shown that on \(L\) and \(R\) the expression \(arccot\left(\frac{x}{y}\right) = \) Arg\(z\).\newline
\linebreak
\noindent Now differentiating the real and imaginary parts partially with respect to both \(x\) and \(y\) using the chain rule and given derivative of \(arccot\), we obtain as previously seen:
\[
\frac{\partial u(x,y)}{\partial x} = \frac{\frac{1}{2} \cdot 2x\cdot \left(x^2+y^2\right)^{\frac{-1}{2}}}{\sqrt{x^2+y^2}} = \frac{x}{x^2+y^2}
\]
\[
\frac{\partial u(x,y)}{\partial y} = \frac{\frac{1}{2} \cdot 2y\cdot \left(x^2+y^2\right)^{\frac{-1}{2}}}{\sqrt{x^2+y^2}} = \frac{y}{x^2+y^2}
\]
\[
\frac{\partial v(x,y)}{\partial x} = \frac{-1}{1 + \left(\frac{x}{y}\right)^2}\cdot \left(\frac{1}{y}\right) = \frac{-y}{x^2+y^2}
\]
\[
\frac{\partial v(x,y)}{\partial y} = \frac{-1}{1 + \left(\frac{x}{y}\right)^2}\cdot \left(\frac{-x}{y^2}\right) = \frac{x}{x^2+y^2}
\]
\noindent Note that as \(y>0\) all of these terms will be defined.\newline
\linebreak
\noindent So the Cauchy-Riemann equations:
\[
\frac{\partial u}{\partial x} = \frac{\partial v}{\partial y} \quad \text{and} \quad \frac{\partial u}{\partial y} = -\frac{\partial v}{\partial x}
\]
have been satisfied by \(f\) on the open region \(I\) and hence \(f\) satisfies the Cauchy Riemann equations on \(U\) as \(U = L \cup R \cup I.
 \quad \Box\) \newline
\linebreak
 \noindent \textbf{Question 3 (5 marks)} Let \(f : \mathbb{C} \to \mathbb{C}\), and define \(g(z) := \overline{f(\overline{z})}\) for every \(z \in \mathbb{C}\). Suppose that \(f\) is complex-differentiable at some \(c \in \mathbb{C}\). Prove from the definition that g is complex-differentiable at \(\overline{c}\).\newline
\linebreak
\textbf{Solution:} \newline 
\linebreak
By the complex-differentiability of \(f\) at \(c \in \mathbb{C}\) we have:
\[
\lim_{z \to c} \frac{f(z)-f(c)}{z-c} \to L = f^{\prime}(c)
\]

\noindent To show that g is complex-differentiable at \(\overline{c}\), we need to show:
\[
\lim_{z \to \overline{c}} \frac{g(z)-g(\overline{c})}{z-\overline{c}} = g^{\prime}({\overline{c}}) \,\, \text{exists}
\]

\noindent Given \(g(z) := \overline{f(\overline{z})}: \)
\[
g(z) - g(\overline{c}) =  \overline{f(\overline{z})} - \overline{f(\overline{\overline{c}})} = \overline{f(\overline{z})} - \overline{f(c)} 
\]

\noindent So we seek to evaluate the limit:
\[
\lim_{z \to \overline{c}} \frac{\overline{f(\overline{z})} - \overline{f(c)} }{z-\overline{c}}
\]

\[
= \lim_{z \to \overline{c}} \overline{\left(\frac{{f(\overline{z})} - {f(c)} }{\overline{z-\overline{c}}}   \right)}
\]

\[
= \lim_{z \to \overline{c}} \overline{\left(\frac{{f(\overline{z})} - {f(c)} }{\overline{z}- c}   \right)}
\]

\noindent Let us make the substitution, \(u = \overline{z}  \implies z = \overline{u} \). \newline
\linebreak
\noindent By the continuity of the conjugate function, as \(z \to \overline{c}\), we have \(u \to c \).\newline
\linebreak
So we now have:
\[
\lim_{u \to c} \overline{\left(\frac{{f(u)} - {f(c)} }{u- c}   \right)}
\]
\noindent Now by the continuity of the conjugate function again, we have:
\[
\overline{\left[ \lim_{u \to c} {\left(\frac{{f(u)} - {f(c)} }{u- c}   \right)}\right]}
\]
\[
= \overline{f^{\prime}\!\left(c\right)} \in \mathbb{C}
\]

\noindent So we have shown that g is complex differentiable at \(\overline{c}\) (and furthermore has derivative, \(g^{\prime}(\overline{c}) = \overline{f^{\prime}(c)}\)). \(\quad \Box\)\newline
\linebreak

\noindent \textbf{Question 4 (4+2 marks)} Consider the power series \(\sum_{n=1}^{\infty} \frac{\left(z-2\right)^n}{n}\).
\begin{enumerate}[(i)]
    \item Determine the radius of convergence \(R\) of this power series. You should find that \(R > 0\).
    \item The general theory learned so far then tells us that this power series defines a holomorphic function \( \lambda \) on \(D(2;R)\). Determine \(\lambda(x)\) for \(x \in \left(2-R, 2 + R\right)\)
\end{enumerate}

\noindent \(\textbf{Solution (i):}\) To determine the radius of convergence \(R\) of the power series
\[ \sum_{n=1}^{\infty} \frac{\left(z-2\right)^n}{n}\] 
We take:
\[
\mathscr{C} := \left\{ r \in \left[ 0, \infty \right) : \text{the sequence} \left( \frac{r^n}{n} \right)_{n=1}^{\infty} \text{ is bounded} \right\}
\]
\linebreak  
\[ \text{If }r \in \left[0,1\right] \text{ then } 0\leq r^n \leq 1 \quad \forall n\in\mathbb{N} \]
\[
\implies 0 \leq \frac{r^n}{n} \leq \frac{1}{n} \leq 1\]
\[\]
So, if \(r \in \left[0,1\right]\) then \(\frac{r^n}{n}\) is bounded by \(\left[0,1\right]\) and hence the interval \(\left[0,1\right]\) is included in \(\mathscr{C}\) \newline
\linebreak
\noindent We can intuitively see that for \(r > 1, \frac{r^n}{n}\) will be unbounded as the exponential numerator will rapidly outgrow the linear denominator. \newline
\linebreak
\noindent Let us prove this formally by trying to contradict this statement and show the sequence is bounded.
\[
\text{If } \frac{r^n}{n} \text{ is bounded then } \exists M \in \mathbb{R} \text{  s.t.  } \forall n \in \mathbb{N}, \, \left| \frac{r^n}{n} \right| \leq M
\]

\[
\text{For } r > 1, n \in \mathbb{N}, \frac{r^n}{n} > 0 \implies \frac{r^n}{n} \leq M 
\]

\[
\implies r^n \leq nM
\]

\[\text{Set } r := 1 + \alpha \text{ so }\alpha > 2\]



\[\text{Then }r^n = \left(1+\alpha\right)^n > \frac{n(n-1)}{2}\alpha^2 \quad \forall \, \alpha > 2 \text{ and }n \in \mathbb{N}\]

\noindent It is again trivial from this inequality that \(\frac{r^n}{n} > \left(n-1\right)\cdot\frac{\alpha^2}{2}\) and hence \(\frac{r^n}{n}\) is unbounded. However, again let us be slightly more formal. \newline
\linebreak
\noindent If we can show that there exists an \(n\) such that \(\frac{n(n-1)}{2}\alpha^2 > nM \) then we will have a contradiction to the statement that the sequence is bounded.\newline
\linebreak
Rearranging we get:
\[n > \frac{2M}{\alpha^2} + 1 = \frac{2M}{\left(r-1\right)^2} + 1 \quad \left(n \in \mathbb{N}\right)\]

\noindent So for whatever \( r > 1 \) we take, then for whatever bound, \(M\) you try to place on the sequence, you can find a term in the sequence such that \(\frac{r^n}{n}\) is greater than M.

\[
\text{Hence }  \frac{r^n}{n} \text{ is unbounded.}
\]

\[
\text{So we get } \mathscr{C} = \left\{ r \in [0, 1] \right\} \implies R = 1 \quad \Box
\]
\hrulefill\newline
\linebreak
\noindent Alternatively, we could use the ratio test result obtained in tutorial 7, question 3 and compute:
\[
R = lim_{n \to \infty} \frac{\left| a_n \right|}{\left| a_{n+1} \right|}
\]
\[
=  lim_{n \to \infty}  \frac{\frac{1}{n}}{\frac{1}{n+1}}
\]
\[
= lim_{n \to \infty} \frac{n+1}{n} = 1
\]
\hrulefill
\newline
\linebreak
\noindent \(\textbf{Solution (ii):}\) We now find \( \lambda(x) \text{ for } x \in \left(2-R, 2+ R\right) = \left(1,3\right)\):\newline
\linebreak
\noindent Since the power series has radius of convergence \(1\), it defines a holomorphic function on \(D(2;1)\) 
\[\lambda(z) := \sum_{n=1}^{\infty} \frac{\left(z-2\right)^n}{n} = \left(z-2\right) + \frac{\left(z-2\right)^2}{2} + \frac{\left(z-2\right)^3}{3}\dots \qquad \textbf{(1)}\] 
\noindent So for \(z \in D(2;1), \)
\[\lambda^{\prime}(z) = 1 + z -2 + \left(z-2\right)^2 + \cdots =  S = 1 + \sum_{n=1}^{\infty} \left(z-2\right)^n \]
\noindent Notice this is a geometric series as \(\left| z-2 \right| < 1 \), so we have:
\[
S = 1 + \frac{z-2}{1-\left(z-2\right)} = 1+ \frac{z-2}{3-z}
\]
\[
=\frac{3-z}{3-z} + \frac{z-2}{3-z} = \frac{1}{3-z}
\]
\linebreak
\noindent Restricting \( \lambda \) to \(\mathbb{R}\) we have \(x \in D(2;1) \cap \mathbb{R} = \left( 1,3 \right) \)

\[\implies \lambda(x) = \sum_{n=1}^{\infty} \frac{\left(x-2\right)^n}{n} = \left(x-2\right) + \frac{\left(x-2\right)^2}{2} + \frac{\left(x-2\right)^3}{3}\dots \in \mathbb{R} \]

\noindent So \(\lambda : \left(1,3\right) \to \mathbb{R}\) is a real valued function with:
\[
\lambda^{\prime}(x) = \frac{1}{3-x}
\]

\[
\implies \lambda(x) = -ln\left|3-x\right| + C \quad \left(x \in \left(1,3\right)\right)
\]
\linebreak
\noindent Notice that from \(\textbf{(1)}\) we have: \(\lambda(2) = 0 = -ln\left|1\right| + C = C \implies C = 0\)

\[
\implies \lambda(x) = -ln\left|3-x\right| \quad \text{ for } x \in \left(1,3\right)
\]
\[\]
\hrulefill

\end{document}

